\chapter{Pendahuluan}
\label{chap:pendahuluan}

\section{Latar Belakang}
\label{sec:latar belakang}

Seiring dengan perkembangan zaman, perkembangan internet di Indonesia sudah semakin maju.  Banyak orang sudah menggunakan fasilitas internet untuk berbagai macam kebutuhan. Contoh dari penggunaan internet adalah untuk mencari informasi, \textit{email}, bermain jejaring sosial, \textit{Internet Banking}, \textit{online shop}, dan lain lain. Menurut Kominfo penggunaan internet di Indonesia sudah mencapai 82 juta orang, delapan puluh persen diantaranya adalah remaja\footnote{Kominfo bint005 , Pengguna Internet di Indonesia Capai 82 Juta, \url{http://kominfo.go.id/index.php/content/detail/3980/Kemkominfo\%3A+Pengguna+Internet+di+Indonesia+Capai+82+Juta/0/berita_satker}, pada tanggal 15 April 2015 pukul 12.58}. Hal ini menunjukkan bahwa internet sudah tidak asing lagi untuk masyarakat di Indonesia ini. Sebagai informasi tambahan bahwa pengguna internet di Indonesia 95 persennya digunakan untuk sosial media atau jejaring sosial \footnote{Kominfo bint005 , Pengguna Internet di Indonesia 63 Juta Orang, \url{http://kominfo.go.id/index.php/content/detail/3415/Kominfo+\%3A+Pengguna+Internet+di+Indonesia+63+Juta+Orang/0/berita_satker}, pada tanggal 15 April 2015 pukul 13.10}.

Twitter adalah salah satu layanan jejaring sosial yang memungkinkan pengguna mem\textit{posting} pesan berbasis teks hingga 140 karakter. Pengguna Twitter menyebutnya sebagai \textit{tweet}. \textit{Tweet} ini akan meneruskan pesan singkat yang ditujukan ke semua \textit{follower} suatu akun\footnote{Dusty Reagan, \textit{Twitter Application Development For Dummies}, Wiley, 2010, halaman 7}.\textit{Follow} adalah salah satu istilah dalam Twitter yang bertujuan untuk mengikuti aktivitas \textit{tweet} suatu akun. Sedangkan cara seseorang untuk dapat memberi rujukan kepada akun Twitter yang lainnya adalah dengan cara melakukan \textit{reply} atau lebih dikenal dengan nama \textit{mention}\footnote{Dusty Reagan, \textit{Twitter Application Development For Dummies}, Wiley, 2010, halaman 9}. Sebagai contoh, diketahui akun bernama @kviniink mem-\textit{follow} @infobdg untuk mengetahui perkembangan apa saja yang tejadi di Kota Bandung. Lalu akun @kviniink ingin bertanya tentang info \textit{mall} yang sedang ramai dikunjungi di Bandung, maka akun @kviniink membuat \textit{mention tweet} kepada akun @infobdg yang berisikan "@infobdg Halo saya ingin bertanya \textit{mall} apa yang sedang ramai dikunjungi di Bandung yah?". Setelah akun @infobdg membaca \textit{mention} dari @kviniink, akun @infobdg melakukan balasan kepada akun @kviniink untuk membalas pertanyaan yang diajukan oleh akun @kviniink. \textit{Tweet} tersebut berisikan "@kviniink mungkin anda bisa mengunjungi PVJ dan Ciwalk".

Transportasi publik sudah banyak digunakan oleh kebanyakan orang di dunia. Bukan hanya di Indonesia saja transportasi publik ini sudah banyak digunakan di luar negeri. Menurut data, angkutan umum di Kota Bandung pada tahun 2013 jumlahnya sudah lebih dari 12000 unit kendaraan\footnote{Oris Riswan , Wow... Jumlah Angkot di Bandung hampir 12 Ribu Unit, \url{http://news.okezone.com/read/2013/11/17/526/898175/wow-jumlah-angkot-di-bandung-hampir-12-ribu-unit}, pada tanggal 15 April 2015 pukul 13.15}. Keuntungan memakai transportasi publik sudah banyak dirasakan di seluruh dunia yaitu untuk mengatasi kemacetan dan mengurangi pemanasan global. Seiring dengan perkembangan teknologi, menaiki transportasi publik menjadi semakin mudah. Dengan adanya KIRI di Indonesia terutama di Kota Bandung, masyarakat dapat manaiki  transportasi publik tanpa harus mengetahui terlebih dahulu jalur yang harus ditempuh pengguna. Pengguna hanya perlu tahu tempat asal dan tempat tujuan untuk menaiki transportasi publik di Kota Bandung.

KIRI API adalah aplikasi pihak ketiga yang memungkinkan pengembang perangkat lunak mendapatkan data tentang info jalur transportasi publik. KIRI API dapat mencari suatu nama lokasi tempat terkenal seperti \textit{mall} dan universitas, selain itu KIRI juga dapat mencari berdasarkan nama jalan seperti jalan astana anyar, jalan pajajaran, jalan kelenteng, dan lain lain. Twitter API adalah aplikasi pihak ketiga yang memungkinkan \textit{programmer} melakukan manipulasi dan pengolahan data di Twitter. Twitter API ini dibagi menjadi dua yaitu REST API dan Streaming API. Dengan memanfaatkan KIRI API dan Twitter API peneliti membuat \textit{Twitter bot} yang dapat membalas \textit{tweet} untuk mencari jalur transportasi publik. \textit{Twitter bot} yang dibuat bersifat \textit{real time} sehingga jika seseorang melakukan mention kepada akun \textit{Twitter bot} maka \textit{Twitter bot} akan menangkapnya dan membalas \textit{mention} tersebut berupa jalur yang harus ditempuh. Contoh dari jalannya \textit{Twitter bot} adalah ketika akun bernama @kviniink melakukan \textit{mention} kepada @kiriupdate untuk bertanya jalur transportasi publik "@kiriupdate bip to ip". Maka Twitter bot @kiriupdate akan menerima \textit{tweet} dari akun @kviniink lalu \textit{tweet} tersebut akan diolah oleh server dan akan di-\textit{reply} dengan tiga buah \textit{tweet}  yaitu 
\begin{enumerate}
	\item "@kviniink Walk about 135 meter from your starting point to Jalan Aceh.",
	\item "@kviniink Take angkot Ciroyom - Antapani at Jalan Aceh, and alight at Jalan Pajajaran about 3.6 kilometer later.",
	\item "@kviniink Walk about 93 meter from Jalan Pajajaran to your destination.".
\end{enumerate}

Dikarenakan \textit{Tweet} memiliki keterbatasan 140 karakter maka \textit{tweet} akan dibagi sesuai dengan instruksi yang dikirimkan dari KIRI API.

Oleh karena itu, dalam penelitian ini dibangun sebuah perangkat lunak yang dapat memudahkan pengguna dalam mencari jalur transportasi publik. Sebuah perangkat lunak yang menggabungkan jejaring sosial Twitter dengan KIRI API. Pengguna dapat melakukan \textit{tweet} kepada \textit{Twitter bot} dengan format yang sudah ditentukan untuk mendapatkan \textit{tweet} yang berisikan rute jalan yang harus ditempuhnya dengan menaiki transportasi publik.

\section{Rumusan Masalah}
Mengacu kepada deskripsi yang diberikan, maka rumusan masalah pada penelitian ini adalah:
\begin{enumerate}
	\item Bagaimana membuat \textit{Twitter bot} untuk mencari jalur transportasi publik?
	\item Bagaimana membuat \textit{Twitter bot} untuk dapat merespon secara \textit{real time}?
	\item Bagaimana mem-\textit{format} petunjuk rute perjalanan dalam keterbatasan \textit{tweet} 140 karakter?
\end{enumerate}

\section{Tujuan}
Tujuan dari penelitian ini adalah:
\begin{enumerate}
	\item Membuat aplikasi \textit{Twitter bot} untuk mencari jalur transportasi publik.
	\item Membuat aplikasi Twitter yang bekerja secara \textit{real time}.
	\item Membuat algoritma untuk memecah instruksi dari KIRI API dan mengubahnya ke dalam bentuk \textit{tweet}.
\end{enumerate}

\section{Batasan Masalah}
Pada pembuatan perangkat lunak ini, masalah-masalah yang ada akan dibatasi menjadi:
\begin{enumerate}
	\item Masukan hanya mencakup Kota Bandung saja.
	\item Masukan yang diinputkan harus benar, memiliki asal dan tujuan yang jelas di Kota Bandung.
	\item Hasil yang dikeluarkan berupa \textit{tweet} jalur transportasi publik.
	\item Media transportasi publik yang digunakan adalah angkutan umum.
	\item Pencarian jalur memanfaatkan KIRI API.
\end{enumerate}

\section{Metode Penelitian}
Pada perangkat lunak yang dibuat ini digunakan beberapa metode dalam penyelesaian masalah yang menjadi topik pada penelitian ini, antara lain:
\begin{enumerate}
	\item Melakukan studi literatur, antara lain:
	\begin{itemize}
		\item KIRI API,
		\item REST API Twitter (\url{https://dev.twitter.com/docs/api/1.1}),
		\item Streaming API Twitter (\url{https://dev.twitter.com/docs/api/streaming}).
	\end{itemize}
	\item Mempelajari pembuatan server dalam bahasa Java.
	\item Melakukan analisis terhadap teori-teori yang sudah dipelajari, guna membangun perangkat lunak yang dimaksud.
	\item Melakukan pengujian terhadap \textit{system} yang sudah dibangun.
	\item Membuat dokumen.
	\item Membuat kesimpulan.
\end{enumerate}

\subsection{Sistematika Pembahasan}

Sistematika pembahasan dalam penelitian ini berupa:
\begin{itemize}
	\item Bab Pendahuluan
	Bab 1 berisi latar belakang, rumusan masalah, tujuan penelitian, batasan masalah, metodologi penelitian, dan sistematika pembahasan.
	\item Bab Dasar Teori
	Bab 2 berisi mengenai Twitter, Twitter API, OAuth, JSON, KIRI API, dan Twitter4J.
	\item Bab Analisis
	Bab 3 berisi analisis meliputi analisis Twitter API, analisis OAuth, analisis KIRI API, analisis Twitter4J, use case, dan rancangan awal diagram kelas.
	\item Bab Perancangan
	Bab 4 berisi tahapan penjelasan diagram kelas lengkap, sequence diagram, dan perancangan antarmuka.
	\item Bab Implementasi dan Pengujian
	Bab 5 berisi tahapan implementasi pada perangkat lunak meliputi tampilan antarmuka perangkat lunak, dan pengujian perangkat lunak.
	\item Bab Kesimpulan dan Saran
	Bab 6 berisi kesimpulan serta beberapa saran untuk pengembangan lebih lanjut dari penelitian yang dilakukan dan perangkat lunak yang dibangun.
\end{itemize}