\chapter{Kesimpulan dan Saran}
\label{chap:kesimpulan dan saran}

Bab ini berisi kesimpulan dan saran dari penelitian yang dilakukan.

\section{Kesimpulan}
Berikut ini adalah kesimpulan yang diambil oleh penulis berdasarkan penelitian yang telah dilakukan:

\begin{enumerate}
	\item Pengguna melakukan \textit{tweet} dengan format lokasi awal \textit{to} lokasi tujuan kepada \textit{Twitter bot}.
	\item \textit{Twitter bot} sudah berjalan dengan lancar dan sudah dapat diimplementasikan kepada akun @kiriupdate. \textit{Twitter bot} dapat menerima \textit{tweet} yang di-\textit{mention} kepada akun \textit{Twitter bot} dan sudah dapat melakukan \textit{reply} dengan benar sesuai hasil yang diberikan oleh KIRI API.
	\item Akses internet mempengaruhi performansi \textit{Twitter bot}.
\end{enumerate}

\section{Saran}
Berdasarkan hasil kesimpulan yang telah dipaparkan, penulis memberi saran sebagai berikut:

\begin{enumerate}
	\item Ketika pengguna ingin mencari lokasi jalan, penulisan nama jalan harus lengkap. Sebagai contoh adalah jalan mekar wangi, jalan kopo. Jika penulisan hanya mekar wangi atau kopo saja, maka terjadi kemungkinan pencarian lokasi tidak akan ditemukan.
	\item Membuat akun \textit{Twitter bot} menjadi premium agar tidak mengalami adanya keterbatasan \textit{tweet} per harinya.
	\item Pada biodata informasi akun \textit{Twitter bot}, sebaiknya diberitahu cara penggunaan \textit{Twitter bot} untuk mencari jalur transportasi publik agar pengguna bisa langsung mencoba.
	\item Pengguna tidak harus melakukan \textit{mention} kepada akun \textit{Twitter Bot}, pengguna dapat memanfaatkan fungsi \textit{hashtag} yang telah diberikan Twitter.
	\item Mengatasi format \textit{tweet} agar tidak terlihat seperti \textit{spam}.
\end{enumerate}