\chapter{Kesimpulan dan Saran}
\label{chap:kesimpulan dan saran}

Bab ini berisi kesimpulan dan saran dari penelitian yang dilakukan.

\section{Kesimpulan}
Berikut ini adalah kesimpulan yang diambil oleh penulis berdasarkan yang telah dilakukan:

\begin{enumerate}
	\item Pengguna melakukan tweet dengan format lokasi awal \textit{to} lokasi tujuan kepada \textit{Twitter bot}.
	\item Twitter bot sudah berjalan dengan lancar dan sudah dapat diimplementasikan kepada user @kiriupdate. Twitter bot dapat menerima tweet yang di mention kepada akun \textit{Twitter bot} dan sudah dapat melakukan reply dengan benar sesuai hasil yang diberikan oleh KIRI API.
	\item Akses internet mempengaruhi performansi \textit{Twitter bot}.
\end{enumerate}

\section{Saran}
Berdasarkan hasil kesimpulan yang telah dipaparkan, penulis memberi saran sebagai berikut.

\begin{enumerate}
	\item Ketika pengguna ingin mencari lokasi jalan, penulisan harus lengkap. Sebagai contoh adalah jalan mekar wangi, jalan kopo. Jika penulisan hanya kopo saja maka pencarian lokasi tidak akan ditemukan.
	\item Membuat akun \textit{Twitter bot} menjadi premium agar tidak mengalami adanya keterbatasan tweet per harinya.
	\item Membuat format \textit{tweet} baru agar lebih mudah dimengerti pengguna.
	\item Pengguna tidak harus melakukan \textit{mention} kepada akun \textit{Twitter Bot}, pengguna dapat memanfaatkan fungsi \textit{hastag} yang telah diberikan Twitter.
	\item Mengatasi format \textit{tweet} agar tidak terlihat seperti \textit{spam}.
\end{enumerate}