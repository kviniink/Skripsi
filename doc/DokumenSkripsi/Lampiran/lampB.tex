\chapter{Kode Program Kelas KIRIGateway}
\label{Kode Program Kelas KIRIGateway}

\begin{lstlisting} [caption= KIRIGateway.java]
{ 
		import java.text.DateFormat;
		import java.text.SimpleDateFormat;
		import twitter4j.*;

		import java.util.ArrayList;
		import java.util.Arrays;
		import java.util.Date;
		import java.util.logging.Level;

		public final class TwitterGateway implements StatusListener{
				public static final String screenName = "@KvinIink";
				private String user;
				private String location[];
				private String latlon[] = new String[2];
				private RoutingResponse routingResponse;
				private Step[] step;
				private Steps steps;
				DateFormat dateFormat = new SimpleDateFormat("HH:mm:ss");
				Date date = new Date();
				
				@Override
				public void onStatus(Status status) {
						user = status.getUser().getScreenName();
						String mentionStatus = status.getText();
						System.out.println("@" + user + " - " + mentionStatus);
						String paramScreenName = screenName.toLowerCase();
						mentionStatus = mentionStatus.toLowerCase().replace(paramScreenName, "");
						location = mentionStatus.split(" to ");
						boolean statusLocation1 = false;
						boolean statusLocation2 = false;
						
						
						if(location.length == 2){
								try {
										System.out.println("Lokasi 1 : "+location[0]);
										System.out.println("Lokasi 2 : "+location[1]);
										//string destination menampung hasil dari JSONObject hasil pencarian apa saja yang ditemukan dari KIRIGateway.getLatLong
										String destination1 = KIRIGateway.GetLatLong(location[0]);
										String destination2 = KIRIGateway.GetLatLong(location[1]);
										
										//dimasukan ke JSONObject 
										JSONObject objDest1 = new JSONObject(destination1);
										JSONObject objDest2 = new JSONObject(destination2);
										
				//memasukan hasil pencarian pertama dari JSONObject ke abribut routingResponse
										JSONObject res1 = objDest1.getJSONArray("searchresult").getJSONObject(0);
										String hasilDest1 = res1.getString("placename");
										latlon[0] = res1.getString("location");
										if(hasilDest1 != null)
										{
												statusLocation1 = true;
										}
										
										JSONObject res2 = objDest2.getJSONArray("searchresult").getJSONObject(0);
										String hasilDest2 = res2.getString("placename");
										latlon[1] = res2.getString("location");
										if(hasilDest2 != null)
										{
												statusLocation2 = true;
										}
										
										//Mendapatkan hasil pencarian lalu dimasukan ke JSONArray paramSteps untuk dipisah-pisah lalu dimasukan ke RoutingResponse
										String hasilPencarian = KIRIGateway.GetTrack(latlon[0], latlon[1]);
										JSONObject objTrack = new JSONObject(hasilPencarian);
										JSONObject routingresults = objTrack.getJSONArray("routingresults").getJSONObject(0);
										JSONArray paramSteps = routingresults.getJSONArray("steps");
										//buat variable step, steps, dan routing response
										step = new Step[paramSteps.length()];
										for (int i = 0; i < step.length; i++) {
												step[i] = new Step(paramSteps.getJSONArray(i).getString(3) + "");
										}
										steps = new Steps(step);
										
										routingResponse = new RoutingResponse(objTrack.getString("status"), steps);
										if(routingResponse.getStatus().equals("ok")){
												for (int i = 0; i < routingResponse.getRoutingResult().getSteps().length ; i++) {
														date = new Date();
														Tweet(user, routingResponse.getRoutingResult().getSteps()[i].getHumanDescription());
												}
												Tweet(user, "Untuk lebih lengkap silahkan lihat di http://kiri.travel?start=" + location[0].replace(" ", "%20") + "&finish="+ location[1].replace(" ", "%20") + "&region=bdo");
												
												for (int i = 0; i < routingResponse.getRoutingResult().getSteps().length ; i++) {
														System.out.println("@"+user + " " + routingResponse.getRoutingResult().getSteps()[i].getHumanDescription());
												}
												System.out.println("@"+user + " Untuk lebih lengkap silahkan lihat di http://kiri.travel?start=" + location[0].replace(" ", "%20") + "&finish="+ location[1].replace(" ", "%20") + "&region=bdo");
										}else{
												System.out.println("status error");
										}
								} catch (Exception ex) {
										try {
												if(!statusLocation1)
												{
														date = new Date();
														Tweet(user,location[0] + " tidak ditemukan");
														System.out.println("@"+user+" "+location[0] + " tidak ditemukan");
												}
												else if(!statusLocation2)
												{
														date = new Date();
														Tweet(user,location[1] + " tidak ditemukan");
														System.out.println("@"+user+" "+location[1] + " tidak ditemukan");
												}
												else
												{
														Tweet(user,"Pencarian tidak ditemukan");
														System.out.println("@"+user+" Pencarian tidak ditemukan");
												}
												//java.util.logging.Logger.getLogger(TwitterGateway.class.getName()).log(Level.SEVERE, null, ex);
										} catch (TwitterException ex1) {
												java.util.logging.Logger.getLogger(TwitterGateway.class.getName()).log(Level.SEVERE, null, ex1);
										}
								}
						}
						
				}

				@Override
				public void onDeletionNotice(StatusDeletionNotice statusDeletionNotice) {
						System.out.println("Got a status deletion notice id:" + statusDeletionNotice.getStatusId());
				}

				@Override
				public void onTrackLimitationNotice(int numberOfLimitedStatuses) {
						System.out.println("Got track limitation notice:" + numberOfLimitedStatuses);
				}

				@Override
				public void onScrubGeo(long userId, long upToStatusId) {
						System.out.println("Got scrub_geo event userId:" + userId + " upToStatusId:" + upToStatusId);
				}

				@Override
				public void onException(Exception ex) {
						System.out.println("onException");
						ex.printStackTrace();
				}

				@Override
				public void onStallWarning(StallWarning sw) {
						System.out.println("onStallWarning");
						throw new UnsupportedOperationException("Not supported yet."); //To change body of generated methods, choose Tools | Templates.
				}
				
				public void Tweet(String user, String paramStatusUpdate) throws TwitterException
				{
						Twitter twitter = new TwitterFactory().getInstance();
						int userLength = user.length();
						int maxTweet = 140 - (userLength + 2 + 9); // ditambah 2 untuk @ dan " ", ditambah 9 untuk waktu " HH:mm:ss";
						
						String[] tampung = paramStatusUpdate.split(" ", 0); //misahin semua kata jadi array of kata
						
						StatusUpdate[] statusUpdate = new StatusUpdate[(paramStatusUpdate.length() / maxTweet) + 1]; //panjang status dibagi batas maksimal huruf yang diperbolehkan, ditambah 1 karena 130/140 = 0
						String[] tampungStatusUpdate = new String[statusUpdate.length]; //bikin banyaknya array sepanjang tweet yang mau di tweet
						
						int increment = 0;
						int incTampung = 0;
						for (int i = 0; i < tampungStatusUpdate.length; i++) {
								tampungStatusUpdate[i] = "@" + user + " "; //setiap tweet harus dimasukin nama user yang dituju
						}
						
						while(increment < statusUpdate.length){
								
								while(incTampung < tampung.length){
										tampungStatusUpdate[increment] += tampung[incTampung]; //nambahin kata
										if(tampungStatusUpdate[increment].length() >= 131){ //melakukan cek apakah tweet sudah over 140 kata atau belom, 131 soalnya 140 dikurang 9 untuk waktu " HH:mm:ss";
												
												tampungStatusUpdate[increment] = tampungStatusUpdate[increment].substring(0, tampungStatusUpdate[increment].length() - tampung[incTampung].length()); //ngebuang kata terakhir yang ditambahin
												tampungStatusUpdate[increment] += " "+ dateFormat.format(date);
												increment++;
												
												break;
										}else{
												tampungStatusUpdate[increment] += " "; //Jika belum 140 kata maka ditambahkan spasi
										}
										incTampung++;
										if(incTampung >= tampung.length){
												tampungStatusUpdate[increment] += " "+ dateFormat.format(date);
												increment++; //kalo kata-kata sudah habis, keluarin dari statement
										}
								}
						}
						
						for (int i = 0; i < statusUpdate.length; i++) {
								statusUpdate[i] = new StatusUpdate(tampungStatusUpdate[i]);
						}
						try {
								for (int i = 0; i < statusUpdate.length; i++) {
										twitter.updateStatus(statusUpdate[i]);
								}
						} catch (TwitterException ex) {
								twitter.updateStatus("@" + user + " Maaf anda sudah penah melakukan pencarian ini sebelumnya. " + dateFormat.format(date) );
								System.out.println("Error : " + ex);
						}
				}
		}
		
}
\end{lstlisting}