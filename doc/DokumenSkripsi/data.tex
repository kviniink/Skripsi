%_____________________________________________________________________________
%=============================================================================
% data.tex v6 (13-04-2015) \ldots dibuat oleh Lionov - Informatika FTIS UNPAR
%
% Perubahan pada versi 6 (13-04-2015)
% - Perubahan untuk data-data ``template" menjadi lebih generik dan menggunakan
%	tanda << dan >>
%
% Perubahan pada versi sebelumnya
% 	versi 5 (10-11-2013)
% 	- Perbaikan pada memasukkan bab : tidak perlu menuliskan apapun untuk 
%	  memasukkan seluruh bab (bagian V)
% 	- Perbaikan pada memasukkan lampiran : tidak perlu menuliskan apapun untuk
%	  memasukkan seluruh lampiran atau -1 jika tidak memasukkan apapun
%	versi 4 (21-10-2012)
%	- Data dosen dipindah ke dosen.tex agar jika ada perubahan/update data dosen
%   mahasiswa tidak perlu mengubah data.tex
%	- Perubahan pada keterangan dosen	
% 	versi 3 (06-08-2012)
% 	- Perubahan pada beberapa keterangan 
% 	versi 2 (09-07-2012):
% 	- Menambahkan data judul dalam bahasa inggris
% 	- Membuat bagian khusus untuk judul (bagian VIII)
% 	- Perbaikan pada gelar dosen
%_____________________________________________________________________________
%=============================================================================
% 								BAGIAN -
%=============================================================================
% Ini adalah file data (data.tex)
% Masukkan ke dalam file ini, data-data yang diperlukan oleh template ini
% Cara memasukkan data dijelaskan di setiap bagian
% Data yang WAJIB dan HARUS diisi dengan baik dan benar adalah SELURUHNYA !!
% Hilangkan tanda << dan >> jika anda menemukannya
%=============================================================================
%_____________________________________________________________________________
%=============================================================================
% 								BAGIAN I
%=============================================================================
% Tambahkan package2 lain yang anda butuhkan di sini
%=============================================================================
\usepackage{booktabs}
\usepackage[table]{xcolor}
\usepackage{longtable}
\usepackage{amsmath}
%=============================================================================

%_____________________________________________________________________________
%=============================================================================
% 								BAGIAN II
%=============================================================================
% Mode dokumen: menetukan halaman depan dari dokumen, apakah harus mengandung 
% prakata/pernyataan/abstrak dll (termasuk daftar gambar/tabel/isi) ?
% - kosong : tidak ada halaman depan sama sekali (untuk dokumen yang 
%            dipergunakan pada proses bimbingan)
% - cover : cover saja tanpa daftar isi, gambar dan tabel
% - sidang : cover, daftar isi, gambar, tabel (IT: UTS-UAS Seminar 
%			 dan UTS TA)
% - sidang_akhir : mode sidang + abstrak + abstract
% - final : seluruh halaman awal dokumen (untuk cetak final)
% Jika tidak ingin mencetak daftar tabel/gambar (misalkan karena tidak ada 
% isinya), edit manual di baris 439 dan 440 pada file main.tex
%=============================================================================
% \mode{kosong}
% \mode{cover}
% \mode{sidang}
%\mode{sidang_akhir}
\mode{final} 
%=============================================================================

%_____________________________________________________________________________
%=============================================================================
% 								BAGIAN III
%=============================================================================
% Line numbering: penomoran setiap baris, otomatis di-reset setiap berganti
% halaman
% - yes: setiap baris diberi nomor
% - no : baris tidak diberi nomor, otomatis untuk mode final
%=============================================================================
\linenumber{yes}
%=============================================================================

%_____________________________________________________________________________
%=============================================================================
% 								BAGIAN IV
%=============================================================================
% Linespacing: jarak antara baris 
% - single: opsi yang disediakan untuk bimbingan, jika pembimbing tidak
%            keberatan (untuk menghemat kertas)
% - onehalf: default dan wajib (dan otomatis) jika ingin mencetak dokumen
%            final/untuk sidang.
% - double : jarak yang lebih lebar lagi, jika pembimbing berniat memberi 
%            catatan yg banyak di antara baris (dianjurkan untuk bimbingan)
%=============================================================================
\linespacing{single}
% \linespacing{onehalf}
%\linespacing{double}
%=============================================================================

%_____________________________________________________________________________
%=============================================================================
% 								BAGIAN V
%=============================================================================
% Bab yang akan dicetak: isi dengan angka 1,2,3 s.d 9, sehingga bisa digunakan
% untuk mencetak hanya 1 atau beberapa bab saja
% Jika lebih dari 1 bab, pisahkan dengan ',', bab akan dicetak terurut sesuai 
% urutan bab.
% Untuk mencetak seluruh bab, kosongkan parameter (i.e. \bab{ })  
% Catatan: Jika ingin menambahkan bab ke-10 dan seterusnya, harus dilakukan 
% secara manual
%=============================================================================
\bab{ }
%=============================================================================

%_____________________________________________________________________________
%=============================================================================
% 								BAGIAN VI
%=============================================================================
% Lampiran yang akan dicetak: isi dengan huruf A,B,C s.d I, sehingga bisa 
% digunakan untuk mencetak hanya 1 atau beberapa lampiran saja
% Jika lebih dari 1 lampiran, pisahkan dengan ',', lampiran akan dicetak 
% terurut sesuai urutan lampiran
% Jika tidak ingin mencetak lampiran apapun, isi dengan -1 (i.e. \lampiran{-1})
% Untuk mencetak seluruh mapiran, kosongkan parameter (i.e. \lampiran{ })  
% Catatan: Jika ingin menambahkan lampiran ke-J dan seterusnya, harus 
% dilakukan secara manual
%=============================================================================
\lampiran{ }
%=============================================================================

%_____________________________________________________________________________
%=============================================================================
% 								BAGIAN VII
%=============================================================================
% Data diri dan skripsi/tugas akhir
% - namanpm: Nama dan NPM anda, penggunaan huruf besar untuk nama harus benar
%			 dan gunakan 10 digit npm UNPAR, PASTIKAN BAHWA BENAR !!!
%			 (e.g. \namanpm{Jane Doe}{1992710001}
% - judul : Dalam bahasa Indonesia, perhatikan penggunaan huruf besar, judul
%			tidak menggunakan huruf besar seluruhnya !!! 
% - tanggal : isi dengan {tangga}{bulan}{tahun} dalam angka numerik, jangan 
%			  menuliskan kata (e.g. AGUSTUS) dalam isian bulan
%			  Tanggal ini adalah tanggal dimana anda akan melaksanakan sidang 
%			  ujian akhir skripsi/tugas akhir
% - pembimbing: isi dengan pembimbing anda, lihat daftar dosen di file dosen.tex
%				jika pembimbing hanya 1, kosongkan parameter kedua 
%				(e.g. \pembimbing{\JND}{  } ) , \JND adalah kode dosen
% - penguji : isi dengan para penguji anda, lihat daftar dosen di file dosen.tex
%				(e.g. \penguji{\JHD}{\JCD} ) , \JND dan \JCD adalah kode dosen
%
%=============================================================================
\namanpm{Kevin Theodorus Yonathan}{2011730037}
\tanggal{03}{6}{2015}
\pembimbing{\TAB}{}    %Lihat singkatan pembimbing anda di file dosen.tex
\penguji{\CAN}{\LCA} 		%Lihat singkatan penguji anda di file dosen.tex
%=============================================================================

%_____________________________________________________________________________
%=============================================================================
% 								BAGIAN VIII
%=============================================================================
% Judul dan title : judul bhs indonesia dan inggris
% - judulINA: judul dalam bahasa indonesia
% - judulENG: title in english
% PERHATIAN: - langsung mulai setelah '{' awal, jangan mulai menulis di baris 
%			   bawahnya
%			 - Gunakan \texorpdfstring{\\}{} untuk pindah ke baris baru
%			 - Judul TIDAK ditulis dengan menggunakan huruf besar seluruhnya !!
%			 - Gunakan perintah \texorpdfstring{\\}{} untuk baris baru
%=============================================================================

\judulINA{Pembuatan \textit{Twitter Bot} Untuk Mencari Jalur Transportasi Publik}

\judulENG{Developing Twitter Bot for Locating Public Transport Route}

%_____________________________________________________________________________
%=============================================================================
% 								BAGIAN IX
%=============================================================================
% Abstrak dan abstract : abstrak bhs indonesia dan inggris
% - abstrakINA: abstrak bahasa indonesia
% - abstrakENG: abstract in english
% PERHATIAN: langsung mulai setelah '{' awal, jangan mulai menulis di baris 
%			 bawahnya
%=============================================================================

\abstrakINA{Transportasi publik banyak digunakan oleh kebanyakan orang di Indonesia. Keuntungan memakai transportasi publik telah banyak dirasakan yaitu untuk mengatasi kemacetan dan mengurangi pemanasan global. KIRI adalah \textit{website} yang dapat mencari jalur transportasi publik dari lokasi awal menuju lokasi tujuan. Kiri membutuhkan akses internet untuk melakukan pencarian. Seiring dengan perkembangan zaman, perkembangan internet di Indonesia sudah semakin maju. Banyak orang sudah menggunakan fasilitas internet untuk berbagai macam kebutuhan terutama untuk jejaring sosial online. Salah satu jejaring sosial yang sudah banyak digunakan orang-orang adalah Twitter. Twitter adalah salah satu layanan jejaring sosial yang memungkinkan pengguna mem\textit{posting} pesan berbasis teks hingga 140 karakter.

Pada tugas akhir ini penulis membuat perangkat lunak \textit{Twitter bot} untuk mencari jalur transportasi publik. Dalam pembuatan Twitter bot, penulis memanfaatkan KIRI API dan Twitter API. KIRI API digunakan untuk memberi jalur transportasi publik, sedangkan Twitter API digunakan untuk menangkap \textit{tweet} dan membalas \textit{tweet}. \textit{Twitter bot} yang dibangun pada penelitian ini menggunakan bahasa Java dan menggunakan library dari Twitter4J.

Dari hasil pengujian, diperoleh bahwa \textit{Twitter bot} yang dibuat sudah berjalan dengan baik. Pengguna sudah dapat mencari jalur transportasi publik dari suatu lokasi menuju lokasi tujuan dengan melakukan \textit{mention} kepada akun \textit{Twitter bot}. Lalu \textit{Twitter Bot} melakukan balasan kepada pengguna berupa \textit{tweet} yang berisikan jalur transportasi publik yang harus ditempuh.
}

\abstrakENG{
Public transport is widely used by most people in Indonesia. Taking advantage of public transportation has been much felt that to tackle congestion and reduce global warming. KIRI is a website that can search for public transport route from the starting location to the destination location. Along with the times, development of the Internet in Indonesia is more advanced. Many people already use the internet facilities for a variety of needs, especially for online social networking. One of the online social networks are already widely used are Twitter. Twitter is one of the online social networking service that allows users to post text-based messages of up to 140 characters.

In this thesis the author makes Twitter bot software to find public transportation route. In the manufacture Twitter bot, the authors utilize KIRI API and Twitter API. KIRI API is used to provide public transportation route, while the Twitter API is used to capture the tweet and reply tweet. Twitter bot that be built using Java language and use the library from Twitter4J.

From the test results, obtained that the Twitter bot that made already running well. Users are able to search for public transport route from one location to the location of the destination by making mention to a Twitter account bot. Then Twitter Bot reply to users in the form of a tweet that contains the public transportation lines that must be taken.
}

%=============================================================================

%_____________________________________________________________________________
%=============================================================================
% 								BAGIAN X
%=============================================================================
% Kata-kata kunci dan keywords : diletakkan di bawah abstrak (ina dan eng)
% - kunciINA: kata-kata kunci dalam bahasa indonesia
% - kunciENG: keywords in english
%=============================================================================
\kunciINA{Twitter, Rute, Transportasi Publik}

\kunciENG{Twitter, Route, Public Transport}
%=============================================================================

%_____________________________________________________________________________
%=============================================================================
% 								BAGIAN XI
%=============================================================================
% Persembahan : kepada siapa anda mempersembahkan skripsi ini ...
%=============================================================================
\untuk{Dipersembahkan untuk diri sendiri}
%=============================================================================

%_____________________________________________________________________________
%=============================================================================
% 								BAGIAN XII
%=============================================================================
% Kata Pengantar: tempat anda menuliskan kata pengantar dan ucapan terima 
% kasih kepada yang telah membantu anda bla bla bla ....  
%=============================================================================
\prakata{
Puji Syukur penulis kepada Tuhan yang telah memberikan rahmatnya sehingga penulis dapat menyelesaikan skripsi yang berjudul "\textbf{Pembuatan Twitter Bot Untuk Mencari Jalur Transportasi Publik}". Skripsi ini disusun dengan maksud memenihi salah satu prasyarat menyelesaikan pendidikan di Jurusan Teknik Informatika, Fakultas Teknologi Informasi dan Sains, Universitas Katolik Parahyangan. Penulis menyadari bahwa dalam penulisan skripsi ini tidak terlepas dari bantuan dan dukungan berbagai pihak. Oleh karena itu, penulis ingin mengucapkan terima kasih kepada:
\begin{itemize}
	\item Pak Thomas Anung Basuki selaku pembimbing yang telah memberikan banyak masukan untuk skripsi ini sehingga skripsi ini dapat diselesaikan dengan baik. 
	\item Pak Pascal Alfadian atas masukan dan tambahan wawasan untuk skripsi ini sehingga skripsi ini dapat diselesaikan dengan baik. 
	\item Keluarga yang selalu memberi dukungan baik moril maupun materil dan doa.
	\item Ibu Luciana dan Pak Chandra selaku penguji yang sudah memberikan banyak masukan dalam penyusunan skripsi ini.
	\item Segenap teman penulis Yohan Sugiyo, Jovan Gunawan, Samuel Christian, Antonio Yaphiar, Steven Christian, Clara, William Wibisono, Calvin Christian, Edbert Jeremiah, teman-teman MWS yang sudah memberi dukungan dan bantuan dalam penyusunan skripsi ini.
\end{itemize}
Semoga segala bantuan dan dukungan berbagai pihak tersebut mendapat berkat dari Tuhan. Semoga skripsi ini berguna bagi semua orang dan dapat dijadikan bahan pembelajaran. Akhir kata, penulis mohon maaf apabila kesalahan dan kekurangan dalam penulisan skripsi ini.
}
%=============================================================================

%_____________________________________________________________________________
%=============================================================================
% 								BAGIAN XIII
%=============================================================================
% Tambahkan hyphen (pemenggalan kata) yang anda butuhkan di sini 
%=============================================================================
\hyphenation{ma-te-ma-ti-ka}
\hyphenation{fi-si-ka}
\hyphenation{tek-nik}
\hyphenation{in-for-ma-ti-ka}
%=============================================================================


%=============================================================================
